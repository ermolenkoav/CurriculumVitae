%%%%%%%%%%%%%%%%%%%%%%%%%%%%%%%%%%%%%%%
% Wenneker Resume/CV
% LaTeX Template
% Version 1.1 (19/6/2016)
%
% Этот шаблон был загружен с:
% http://www.LaTeXTemplates.com
%
% Оригинальный автор:
% Фритс Веннекер (http://www.howtotex.com) с обширными модификациями от
% Vel (vel@LaTeXTemplates.com) и Алексей Ермоленко (ermolenkoav@gmail.com)
%
% Лицензия:
% CC BY-NC-SA 3.0 (http://creativecommons.org/licenses/by-nc-sa/3.0/
%%%%%%%%%%%%%%%%%%%%%%%%%%%%%%%%%%%%%%

\documentclass[a4paper,11pt]{memoir} % Font and paper size
\input{structure} % Include the file specifying document layout and packages

\userinformation{ % Set the content that goes into the sidebar of each page
    \begin{flushright}

        \includegraphics[width=1\columnwidth]{photo}\\[\baselineskip] % Your photo
        \small % Smaller font size


%----------------------------------------------------------------------------------------
%   КОНТАКТЫ
%----------------------------------------------------------------------------------------

        \textbf{Контакты} \\ % Name
        \href{https://ermolenkoav.t.me/}{Telegram} \\ % Email
        \href{https://www.linkedin.com/in/ermolenkoav/}{LinkedIn} \\ % Your URL
        \href{https://github.com/ermolenkoav}{GitHub} \\
        \href{https://join.skype.com/invite/OaVc1zBWmoRm}{Skype} \\
        \href{mailto:ermolenkoav@gmail.com}{Email} \\ % Email
        +7 (918) 565-36-01 \\ % Your phone number

        \Sep
        \textbf{Адрес} \\
        Ростов-на-Дону, Россия \\

%----------------------------------------------------------------------------------------
%   НАВЫКИ
%----------------------------------------------------------------------------------------
        \Sep

        \textbf{Языки Программирования}
        \begin{itemize}[left=0pt]
            \item Golang
            \item SQL
            \item C++
        \end{itemize}

        \Sep

        \textbf{Базы данных}
        \begin{itemize}[left=0pt]
            \item PostgreSQL
            \item ClickHouse
            \item Redis
            \item MSSQL
            \item MYSQL
            \item SQLite
        \end{itemize}

        \Sep

        \textbf{Фреймворки и инструменты}
        \begin{itemize}[left=0pt]
            \item Docker Compose
            \item Kubernetes
            \item gRPC
            \item Kafka
            \item \href{https://github.com/cilium/ebpf}{Cilium eBPF}
        \end{itemize}
        \vfill % Пробел под этим блоком

    \end{flushright}
}
%----------------------------------------------------------------------------------------
\begin{document}
    \userinformation % Вывод информации в левой колонке
    \framebreak % Конец первой колонки

%----------------------------------------------------------------------------------------
%   ЗАГОЛОВОК
%----------------------------------------------------------------------------------------

    \cvheading{Алексей Ермоленко}
    \cvsubheading{Senior Software Engineer (Golang, SQL)}

%----------------------------------------------------------------------------------------
%   ОБО МНЕ
%----------------------------------------------------------------------------------------

    \aboutme{Обо мне}{Senior Software Engineer, специализирующийся на высоконагруженных системах, обработке больших данных и микросервисной архитектуре. Увлечён оптимизацией производительности и внедрением масштабируемых решений на Golang, SQL и C++.}

%----------------------------------------------------------------------------------------
%   ОПЫТ РАБОТЫ
%----------------------------------------------------------------------------------------

    \CVSection{Опыт работы}

    \CVItem{С Сентября 2024, \textit{Инженер-программист}, ООО Автомакон.}{
        \begin{itemize}[left=0pt]
            \item Осуществлял декомпозиции монолита в сервис-ориентированную архитектуру, оптимизируя управление данными в Postgres.
            \item Автоматизировал инструментарий проверки консистентности данных.
        \end{itemize}
    }

    \CVItem{Октябрь 2022 - Сентябрь 2024, \textit{Инженер-программист}, ООО ВБ Тех.}{
        \begin{itemize}[left=0pt]
            \item Разработал систему финансового анализа, обрабатывающую от 1500 до пиковых 70000 rps c доступностью не ниже 99.99\% в режиме мягкого реального времени. В качестве ядра системы использовался кластер ClickHouse. Для достижения этого результата заменил NATS на Kafka во всех сервисах отдела.
            \item Заменил кластер NATS на кластер Kafka в высоконагруженном проекте.
            \item Участвовал в запуске нового баланса продавцов.
        \end{itemize}
    }

    \CVItem{Октябрь 2021 - Июнь 2022, \textit{Инженер-программист}, ООО Ддос-Гвард.}{
        \begin{itemize}[left=0pt]
            \item Занимался поддержкой и внедрением нового функционала в высоконагруженный реверспрокси, обеспечивал соответствие RFC1945, 2616.
            \item Участвовал в старте разработки первоначального функционала l4 балансировщика нагрузки.
        \end{itemize}
    }

    \CVItem{Февраль 2020 - Октябрь 2021, \textit{Инженер-программист}, ООО Ростпэй.}{
        \begin{itemize}[left=0pt]
            \item Разработал талстый клиент решение для доставки таргетированной рекламы, полностью заменив предшествующие подходы.
            \item Кастомизировал браузер Chromium.
        \end{itemize}
    }

    \CVItem{Ноябрь 2017 - Январь 2020, \textit{Инженер-программист}, Центр Нейротехнологий, ЮФУ.}{
        \begin{itemize}[left=0pt]
            \item Разрабатывал электронные устройства и программное обеспечение управления физиологическими экспериментами.
            \item Разрабатывал электронные устройства электроэнцефалографии, электромиостимуляции и миостимуляции.
        \end{itemize}
    }

    \CVItem{Февраль 2015 - Октябрь 2017, \textit{Инженер-программист}, Don-Strike ltd.}{
        \begin{itemize}[left=0pt]
            \item Разработка IoT-устройств и ПО для физиологической лаборатории.
            \item Написание технической документации.
        \end{itemize}
    }

    \CVItem{Сентябрь 2013 - Февраль 2015, \textit{Младший инженер-программист}, ФГАНУ НИИ Спецвузавтоматика.}{
        \begin{itemize}[left=0pt]
            \item Участвовал в разработке IoT-устройств. Осуществил миграцию с EAGLE на Altium Designer новых проектных работ.
            \item Прототипирование электронных схем, написание технической документации.
        \end{itemize}
    }

%----------------------------------------------------------------------------------------
%	NEW PAGE DELIMITER
%----------------------------------------------------------------------------------------
    \clearpage % Start a new page
    \userinformation % Print your information in the left column
    \framebreak % End of the first column

    \tiny

%----------------------------------------------------------------------------------------
%   ОБРАЗОВАНИЕ
%----------------------------------------------------------------------------------------
    \CVSection{Образование}

    \CVItem{2011 - 2014, Южный Федеральный Университет}{Факультет физики, Аспирантура}

    \CVItem{2006 - 2011, Южный Федеральный Университет}{Факультет физики, Медицинская физика, Специалитет}

    \CVItem{2008 - 2012, Южный федеральный университет}{Экономический факультет, Бакалавр экономики}

    \Sep % Дополнительный пробел после раздела

%----------------------------------------------------------------------------------------
%   НАГРАДЫ
%----------------------------------------------------------------------------------------
    \CVSection{Награды}
    \CVItem{2022, ООО DDoS-Guard}{За креативность и нестандартность мышления в решении поставленных задач}

    \CVItem{2010, \textit{Факультет Физики}, Южный Федеральный Университет}{Лауреат 62й Студенческой Научной Конференции ЮФУ}
    \Sep % Extra whitespace after the end of a section

%----------------------------------------------------------------------------------------
%   КУРСЫ
%----------------------------------------------------------------------------------------
    \CVSection{Курсы}
    \CVItem{2018, Otus}{Разработчик C++}

    \CVItem{2020, Otus}{Spring Framework Developer}
    \Sep % Extra whitespace after the end of a section

%----------------------------------------------------------------------------------------
%   ИНТЕРЕСЫ
%----------------------------------------------------------------------------------------
    \CVSection{Интересы}
    \CVItem{Профессиональные}{Разработка программного обеспечения, разработка электронных устройств}

    \CVItem{Личные}{Пешие прогулки по горам}
    \Sep % Дополнительный пробел после раздела

%---------------------------------------------------------------------------------------
\end{document}